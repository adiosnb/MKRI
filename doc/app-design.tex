\chapter{Návrh aplikácie}
\label{app-design}

	V~tejto kapitole sú popísané jednotlivé prvky firewall aplikácie, hlavne teda grafické prostredie 
	pre správu a analýzu, ale aj jadro samotného firewallu. Celý projekt je vyvinutý v~jazyku \texttt{Python3.6},
	ktorý ponúka univerzálnosť, jednoduchosť a veľa možností, ktoré uľahčujú a zrýchľujú vývoj projektov.

	\section{Virtuálne prostredie}
	\label{pipenv}
		Celá aplikácia je vyvíjaná vo virtuálnom prostredí \texttt{pipenv}, aby bol zaručená zhoda verzií 
		jednotlivých externých modulov na rôznych systémoch, hlavne teda počas vývoja a nasadenia aplikácie
		na strane servera. Všetky potrebné balíky su definované aj s~konkrétnymi verziami v~súbore \texttt{Pipfile}.

	\section{Grafické rozhranie}
	\label{gui}
		Táto aplikácia je navrhnutá, aby bežala ako služba na servery, ktorý slúži ako firewall. Jedným
		z~cieľov tohto projektu je, aby aplikácia poskytovala štatistiky vo forme grafov. Aby sme sa vyhli
		inštalácií grafického prostredia na firewalle, ktoré je závislé na množstve rôznych balíkov, rozhodli
		sme sa vytvoriť užívateľské rozhranie pomocou webového frameworku \texttt{Django} (popisaný v~sekcií
		\ref{django}). Cez tento web bude možné upravovať konfiguráciu firewallu a sledovať aj dynamicky 
		generované grafy pomocou \texttt{HighCharts} (popísaný v~sekcií \ref{highcharts}).

		\subsection{Django}
		\label{django}
			Webový framework \texttt{Django} je napísaný v~jazyku \texttt{Python} a slúži na rýchly vývoj 
			a návrh aplikácií. Hlavnou myšlienkou je nevyvíjať už existujúce veci a riešenia, ale zamerať
			sa hlavne na novú aplikáciu. V~základe každého \texttt{Django} projektu nájdeme autentifikáciu
			užívateľov a abstrakciu nad rôznymi databázovými systémami. Tento framework používa poupravený 
			\texttt{Jinja2} šablónovací systém pre generovanie html stránok \cite{django}.
		
		\subsection{HighCharts}
		\label{highcharts}
			Tento JavaScriptový modul slúži na generovanie veľkého množstva rôznych interaktívnych webových
			grafov ako napríklad čiarové, koláčové či pruhový graf. Práca s~týmto modulom je veľmi jednoduchá
			a spočíva vo vytvorení reťazca dát vo formáte \texttt{JSON}, ktorý obsahuje všetky potrebné 
			informácie. O~zvyšok sa uz modul postará sám \cite{highcharts}.
		        
	\section{Firewall}
		Tento firewall používa \texttt{nftables} na filtrovanie sieťovej trafiky. Konfigurácia môže byť 
		zadaná priamo pomocou programu \texttt{nft} alebo pomocou webového grafického rozhrania, kde je
		textové pole s~aktuálnou konfiguráciou, ktorú možno upravovať. 
		
	\section{Štatistika}
		Ako zdroj dát pre generovanie grafov firewall používa čítače, ktoré sú pridané priamo v~\texttt{nftables}
		konfigurácií. Používa aj niektoré systémové nástroje, ktoré túto štatistiku zbierajú. Tieto dáta sa 
		načítajú a spracujú pre každú požiadavku na webový server, ktorý vygeneruje príslušné \texttt{HighCharts}
		grafy.